\documentclass[english]{article}
\usepackage[T1]{fontenc}
\usepackage[latin9]{inputenc}
\usepackage{color}
\usepackage{amsmath}
\usepackage{amssymb}
\usepackage{esint}
\usepackage{babel}
\usepackage[round]{natbib}
\usepackage{color,hyperref}
\definecolor{darkblue}{rgb}{0.0,0.0,0.3}
\hypersetup{colorlinks,breaklinks,
linkcolor=darkblue,urlcolor=darkblue,
anchorcolor=darkblue,citecolor=darkblue}
\usepackage[capitalise, noabbrev]{cleveref}

\begin{document}
	\title{Solving PDEs Associated with Economic Models}
	\author{\large{\textsc{Matthieu Gomez \thanks{I thank Valentin Haddad, Ben Moll, and Dejanir Silva for useful discussions.}}}}
	\date{\today}
	\maketitle
	This package \href{https://github.com/matthieugomez/EconPDEs.jl}{EconPDEs.jl} introduces a fast and robust way to solve systems of PDEs + algebraic equations (i.e. DAEs) associated with economic models. This note details the underlying algorithm. 
	\paragraph{Difference with \citet{achdou2014heterogeneous}}
	 \citet{achdou2014heterogeneous} focus on linear PDEs of the form
	\begin{align*}
0&=f_1(V)  + f_2(x) \partial_x V  + f_3(x) \partial_{xx} V + \partial_t V
	\end{align*}
	In contrast, the package solves non-linear PDEs of the form
		\begin{align*}
			0&=f_1(V)  + f_2(x, \partial_x V)  + f_3(x, \partial_x V) \partial_{xx} V+ \partial_t V
		\end{align*}

	

	\paragraph{Step 1: Write Finite Difference Scheme}
	The system of PDEs is written on a state space grid and derivatives are substituted by finite difference approximations. As in \citet{achdou2014heterogeneous},  first order derivatives are upwinded. This helps to ensure that the boundary counditions are satisfied, as well as making the scheme monotonous.

	\paragraph{Step 2: Solve Finite Difference Scheme}

	Denote $V_t$  the solution of the PDE. We can alwyays write the HJB is $0 =F(V_t) + \partial_t V$. 
	I propose to solve for $V$ using a fully implicit Euler method.  Updates take the form 
	\begin{align*}
		\forall t \leq T \hspace{1cm} 0&= F(V_{t+1}) + \frac{1}{\Delta}(V_{t+1} -V_{t})
	\end{align*}
	Each time step requires to solve a non-linear equation, which is solved using a Newton-Raphson method. 

	If the Newton-Raphson step is not successful, I decrease $\Delta$ (since this non-linear step converges if the guess is sufficiently close to the solution). If it is successful, I increase $\Delta$, to speed up the algorithm.

	This method is most similar to a method used in the fluid dynamics literature. In this context, it is called the Pseudo-Transient Continuation method, and is denoted $\Psi tc$. Formal conditions for the convergence of the algorithm are given in  \citet{kelley1998convergence}.\par



		
	\bibliography{bib}
	\bibliographystyle{aer}
\end{document}
